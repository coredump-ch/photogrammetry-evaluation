% Table of Contents - List of Tables/Figures/Listings and Acronyms

\refstepcounter{dummy}

\pdfbookmark[1]{\contentsname}{tableofcontents} % Bookmark name visible in a PDF viewer

\setcounter{tocdepth}{2} % Depth of sections to include in the table of contents - currently up to subsections

\setcounter{secnumdepth}{3} % Depth of sections to number in the text itself - currently up to subsubsections

\manualmark
\markboth{\spacedlowsmallcaps{\contentsname}}{\spacedlowsmallcaps{\contentsname}}
\tableofcontents
\automark[section]{chapter}
\renewcommand{\chaptermark}[1]{\markboth{\spacedlowsmallcaps{#1}}{\spacedlowsmallcaps{#1}}}
\renewcommand{\sectionmark}[1]{\markright{\thesection\enspace\spacedlowsmallcaps{#1}}}

\clearpage

\begingroup
\let\clearpage\relax
\let\cleardoublepage\relax
\let\cleardoublepage\relax

%----------------------------------------------------------------------------------------
%	List of Figures
%----------------------------------------------------------------------------------------

%\refstepcounter{dummy}
%\addcontentsline{toc}{chapter}{\listfigurename} % Uncomment if you would like the list of figures to appear in the table of contents
%%\pdfbookmark[1]{\listfigurename}{lof} % Bookmark name visible in a PDF viewer
%
%\listoffigures
%
%\vspace*{8ex}
%\newpage

%----------------------------------------------------------------------------------------
%	List of Tables
%----------------------------------------------------------------------------------------

%\refstepcounter{dummy}
%\addcontentsline{toc}{chapter}{\listtablename} % Uncomment if you would like the list of tables to appear in the table of contents
%\pdfbookmark[1]{\listtablename}{lot} % Bookmark name visible in a PDF viewer
%
%\listoftables
%
%\vspace*{8ex}
%\newpage

%----------------------------------------------------------------------------------------
%	List of Listings
%----------------------------------------------------------------------------------------

%\refstepcounter{dummy}
%\addcontentsline{toc}{chapter}{\lstlistlistingname} % Uncomment if you would like the list of listings to appear in the table of contents
%\pdfbookmark[1]{\lstlistlistingname}{lol} % Bookmark name visible in a PDF viewer
%
%\lstlistoflistings
%
%\vspace*{8ex}
%\newpage

%----------------------------------------------------------------------------------------
%	Acronyms
%----------------------------------------------------------------------------------------

\refstepcounter{dummy}
\addcontentsline{toc}{chapter}{Acronyms} % Uncomment if you would like the acronyms to appear in the table of contents
%\pdfbookmark[1]{Acronyms}{acronyms} % Bookmark name visible in a PDF viewer

\markboth{\spacedlowsmallcaps{Acronyms}}{\spacedlowsmallcaps{Acronyms}}

\chapter*{Glossar}

\begin{acronym}[UML]
	\acro{G-Code}{Code bestehend aus CNC Instruktionen, wird häufig für 3D-Drucker
	einesetzt.}
	\acro{GIS}{Geografische Informationssysteme}
	\acro{GNSS}{Global Navigation Satellite System, \zb{} GPS, GLONASS oder
	Galileo.}
	\acro{HSR}{Hochschule für Technik Rapperswil}
	\acro{Orthofoto}{Eine verzerrungsfreie und massstabsgetreue Abbildung der
	Erdoberfläche, die durch photogrammetrische Verfahren aus Luft- oder
	Satellitenbildern abgeleitet wird.}
	\acro{Photogrammetrie}{Eine Gruppe von Messmethoden und Auswerteverfahren der
	Fernerkundung, um aus Fotografien und genauen Messbildern eines Objektes seine
	räumliche Lage oder dreidimensionale Form zu bestimmen.}
	\acro{SFM}{Structure From Motion, eine Methode um 3D-Modelle aus "<bewegten
	Bildern"> zu erzeugen.}
	\acro{STL}{STereoLithography, ein für 3D-Druck geeignetes Dateiformat.}

\end{acronym}

\endgroup

\cleardoublepage
