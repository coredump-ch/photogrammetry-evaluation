\chapter{Appendix E -- Softwareliste}

\label{ch:software}

{

\setlength{\parindent}{0em}
\setlength{\parskip}{0.8em}

Dieses Kapitel enthält eine alphabetisch sortierte Liste von
photogrammetrie-bezogener Software mit Kosten und Beschreibung.

%%%

\section{Agisoft PhotoScan}

Professionelle Photogrammetrie-Software zur Rekonstruktion von 3D-Modellen aus
Fotos, sowie zur Erstellung von Orthofotos und Oberflächenmodellen.

\textbf{Lizenz:} Proprietär.

\textbf{Kosten:} Pro Version 3'499 USD, Standard Version 179 USD.
Sonderkonditionen für schulische Zwecke.

\textbf{URL:} \url{http://www.agisoft.com/}

%%%

\section{Blender}

Professionelle Open Source Software zur Erstellung, Modellierung, Bearbeitung
und Animation von 3D-Modellen. Weitere Infos in \autoref{workflow:blender}.

\textbf{Lizenz:} GPLv2.

\textbf{Kosten:} Gratis.

\textbf{URL:} \url{https://www.blender.org/}

%%%

\section{CloudCompare}

Open Source Software zum Betrachten und Verarbeiten von Meshes.

\textbf{Lizenz:} GPLv2.

\textbf{Kosten:} Gratis.

\textbf{URL:} \url{http://www.danielgm.net/cc/}

%%%

\section{Cura}

Von der Firma Ultimaker entwickelte Open Source Slicer-Software zum Aufbereiten
von 3D-Modellen für den 3D-Druck.

\textbf{Lizenz:} AGPLv2.

\textbf{Kosten:} Gratis.

\textbf{URL:} \url{https://ultimaker.com/en/products/cura-software}

%%%

\section{GPicSync}

Open Source Software zum Geocodieren von Fotos. Weitere Infos in
\autoref{workflow:hsr:geocoding}.

\textbf{Lizenz:} GPLv2.

\textbf{Kosten:} Gratis.

\textbf{URL:} \url{https://github.com/metadirective/gpicsync}

%%%

\section{GPSBabel}

Software zum Extrahieren von GPS-Daten aus GPS-Geräten, sowie zum Konvertieren
von GPS-Daten zwischen verschiedenen Formaten. Weitere Infos in
\autoref{workflow:hsr:geocoding}.

\textbf{Lizenz:} GPLv2 or later.

\textbf{Kosten:} Gratis.

\textbf{URL:} \url{http://www.gpsbabel.org/}

%%%

\section{Hugin}

Open Source Software zum Zusammenfügen (Stitching) von Panoramas sowie zum
Entzerren von Fotos mittels Linsenprofilen. Weitere Infos in
\autoref{sec:image-correction}.

\textbf{Lizenz:} GPLv2.

\textbf{Kosten:} Gratis.

\textbf{URL:} \url{http://hugin.sourceforge.net/}

%%%

\section{Menci APS}

Professionelle Photogrammetrie-Software zur Rekonstruktion von 3D-Modellen aus
Fotos, sowie zur Erstellung von Orthofotos und Oberflächenmodellen.

\textbf{Lizenz:} Proprietär.

\textbf{Kosten:} Ca. 6000 USD.

\textbf{URL:} \url{http://www.agisoft.com/}

%%%

\section{MeshLab}

Open Source Software zum Betrachten und Verarbeiten von Meshes. Weitere Infos in
\label{workflow:mesh-generating}.

\textbf{Lizenz:} GPLv2.

\textbf{Kosten:} Gratis.

\textbf{URL:} \url{http://sourceforge.net/projects/meshlab/}

%%%

\section{OpenDroneMap}

Open Source Photogrammetrie-Pipeline zur Erstellung von Orthofotos und
Oberflächenmodellen aus Luftbildern. Weitere Infos in \autoref{opendronemap}.

\textbf{Lizenz:} Freie Software, viele unterschiedlich lizenzierte Komponenten.

\textbf{Kosten:} Gratis.

\textbf{URL:} \url{https://github.com/OpenDroneMap/OpenDroneMap}

%%%

\section{Pix4Dmapper Pro}

Professionelle Photogrammetrie-Software zur Rekonstruktion von 3D-Modellen aus
Fotos, sowie zur Erstellung von Orthofotos und Oberflächenmodellen.

\textbf{Lizenz:} Proprietär.

\textbf{Kosten:} Monatlich 320 CHF, jährlich 3'200 CHF oder einmalig 7'900 CHF.
Sonderkonditionen für nichtkommerzielle Zwecke und schulische Zwecke.

\textbf{URL:} \url{https://pix4d.com/}

%%%

\section{VisualSFM}

Photogrammetrie-Software zur Rekonstruktion von 3D-Modellen aus Fotos. Weitere
Infos in \autoref{workflow:vsfm} und \autoref{ch:installing-vsfm}

\textbf{Lizenz:} Quelloffen aber Proprietär. Einzelne Komponenten Freie Software. Nur zu
nichtkommerziellen und schulischen Zwecken einsetzbar.

\textbf{Kosten:} Gratis.

\textbf{URL:} \url{http://ccwu.me/vsfm/}

}
