\chapter{Einleitung}

\label{ch:einleitung}

%----------------------------------------------------------------------------------------

Dieses Kapitel erläutert kurz den Ursprung dieses Projektes sowie dessen Ziele.

%----------------------------------------------------------------------------------------

\section{Das Crowdfunding-Projekt}\label{sec:crowdfunding}

Wir sind Mitglieder des im Herbst 2013 in Rapperswil gegründeten Hackerspaces
"<Coredump">\footnote{\url{https://www.coredump.ch/}}. Unsere Ziele sind der
kreative Umgang mit Technologie sowie der Know-How Austausch an regelmässigen
Treffen. Wir möchten unseren Mitgliedern gute Infrastruktur für technische
Projekte, v.a. im Bereich der Informatik und Elektrotechnik bieten.

\marginpar{Die Schweizer Crowdfunding-Plattform wemakeit wurde im Februar 2012
von der Kulturkommunikatorin Rea Eggli, dem Künstler Johannes Gees und dem
Interaction Designer Jürg Lehni gegründet und ist mittlerweile die grösste
Crowdfunding-Plattform in der Schweiz.}

Um die Herstellung von selbst gestalteten 3D-Teilen zu ermöglichen, starteten
wir im Januar 2015 ein Crowdfunding-Kampagne auf
Wemakeit\footnote{\url{https://wemakeit.com/projects/3d-drucker-fuer-rapperswil/}}
zur Finanzierung eines 3D-Druckers für unseren Verein. Wie es bei solchen
Crowdfunding-Aktionen üblich ist, boten wir verschiedene Belohnungen an, je nach
Unterstützer-Level. Eine davon sollte einen lokalen Bezug haben. Wir entschieden
uns daher dafür, das Schloss Rapperswil als 3D-Modell zu erfassen und in eine
3D-druckbare Form zu bringen.

%----------------------------------------------------------------------------------------

\section{Das 3D-Erfassungs-Projekt}\label{sec:3d-project}

Zuerst wandten wir uns an die Ortsgemeinde Rapperswil-Jona, wo wir Grundrisse
und weitere Architektur-Pläne des Schlosses erhielten. Leider waren diese Pläne
nicht vollständig genug, um daraus ein 3D-Modell des Schlosses zu rekonstruieren.

Da also keine Pläne verfügbar waren, mussten wir den 3D-Umriss des Schlosses
selber erfassen. Unser erster Gedanke war die Vermessung mittels Laserscanning.
Dafür wandten wir uns an Prof. Stefan Keller an der HSR, um von seinem KnowHow
im Bereich Geowissenschaft / Geomatik / GIS zu profitieren.

Prof. Keller schlug uns jedoch vor, das Schloss stattdessen mithilfe einer
ferngesteuerten Kamera-Drohne zu erfassen und mit Pho\-to\-gram\-met\-rie-Software
weiterzuverarbeiten. Wir begannen mit der Recherche-Arbeit in diesem Themenfeld
und so entstand ein Projektplan. Die HSR würde die Finanzierung übernehmen, wir
bieten dem Geometa Lab der HSR dafür im Gegenzug das gesammelte KnowHow in Form
dieser Dokumentation.

%----------------------------------------------------------------------------------------

\section{Ziele}\label{sec:goals}

Die Ziele dieses Projektes wurden wie folgt definiert:

\begin{enumerate}
	\item Erarbeitung eines Workflows zur Erstellen eines 3D-Modells des Schloss
		Rapperswils aus Drohnen-Luftbildern, bevorzugt mithilfe von Open Source
		Software.
	\item Evaluation von verschiedenen Software-Lösungen zur Erstellung einer
		3D-Figur aus einem physischen Objekt mittels Photogrammetrie.
	\item Evaluation von verschiedenen Software-Lösungen zur Erstellung eines
		Orthophotos (entzerrtes 2D-Luftbild) aus Drohnen-Luft\-bil\-dern.
\end{enumerate}

\noindent Diese Dokumentation ist das Ergebnis und erfüllt alle obenstehenden Ziele.
