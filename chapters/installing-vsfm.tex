\chapter{Appendix A -- Installation VisualSFM}

\label{ch:installing-vsfm}

%----------------------------------------------------------------------------------------

Die folgende Anleitung beschreibt die Installation von
VisualSFM\footnote{\url{http://ccwu.me/vsfm/}} (nachfolgend VSFM genannt), auf
einem Computer mit Ubuntu\footnote{\url{http://www.ubuntu.com}} 14.04 LTS
Betriebssystem und einer CUDA-fähigen nVidia Grafikkarte. CUDA-Unterstützung ist 
optional, aber empfohlen.

%----------------------------------------------------------------------------------------

\section{Einleitung}

VSFM ist eine GUI-Applikation für 3D-Rekonstruktion mittels "<Structure from
Motion"> (SFM) Technik. Das Projekt integriert viele Einzelkomponenten welche
meist von Universitäten entwickelt wurden.

\subsection{Abhängigkeiten}

Für den VSFM 3D-Reconstruction Prozess haben wir folgende Bibliotheken verwendet:

\begin{itemize}
	\item SiftGPU\footnote{\url{http://www.cs.unc.edu/~ccwu/siftgpu/}}:
		Eine SIFT\cite{lowe:2004}-Implementation für GPUs.
	\item PMVS (Patch-based Multi-view Stereo)\footnote{\url{http://www.di.ens.fr/pmvs/}}:
		Erstellung einer Point-Cloud aus Kamerabildern.
		Das Resultat enthält gemeinsame Features als
		Punkte mit an der Kamera ausgerichteten Normalen.
	\item CMVS (Clustering Views for Multi-view Stereo)\footnote{\url{http://www.di.ens.fr/cmvs/}}:
		Ergänzt PMVS um Clustering für schnellere Resultate.
	\item PBA\footnote{\url{http://grail.cs.washington.edu/projects/mcba/}}:
		Bündelblockausgleichung (Englisch: Bundle Adjustment) auf mehreren CPUs.
\end{itemize}

\noindent Da für die meisten dieser Komponenten nicht als Ubuntu-Pakete
verfügbar sind, müssen sie manuell kompiliert werden.

\subsection{Alternativen}

VSFM arbeitet mit vielen alternativen SFM-Programmen zusammen oder kann Teile davon selbst
benutzen. Sie sind jedoch nicht direkt für den Betrieb von VSFM nötig und wurden
nicht näher betrachtet:

\begin{itemize}
	\item CMP-MVS\footnote{\url{http://ptak.felk.cvut.cz/sfmservice/websfm.pl?menu=cmpmvs}}
	\item MVE\footnote{\url{http://www.gris.informatik.tu-darmstadt.de/projects/multiview-environment/}}
	\item SURE\footnote{\url{http://www.ifp.uni-stuttgart.de/publications/software/sure/index.en.html}}
	\item MeshRecon\footnote{\url{http://www-scf.usc.edu/~zkang/software.html}}
\end{itemize}

%----------------------------------------------------------------------------------------

\section{Installation}

Mit dem folgenden Installationsscript kann VSFM automatisch auf einem Ubuntu 14.04
installiert werden.

In dieser Version werden alle Abhängigkeiten automatisch aus dem
Internet heruntergeladen. Sie sind ebenfalls auf der (TODO: CD/Stick/Internet) vorhanden 
für den Fall, dass die Ressourcen nicht mehr online verfügbar sind. Es wird empfohlen
nicht das ganze Script auf einmal auszuführen, sondern es Schritt für Schritt durchzugehen.
Möglicherweise haben sich Systemvoraussetzungen in der Zwischenzeit geändert
(z.B. nVidia-Treiber).

\begin{minted}[bgcolor=tango-bg,frame=lines,framesep=2mm,samepage=true,fontsize=\footnotesize]{bash} 
# Whole 2D to 3D reconstruction toolchain setup for Ubuntu 14.04 64bit.
# Make sure to enable the multiverse repository!
# Sources: 
# - http://ccwu.me/vsfm/install.html#linux
# - http://www.10flow.com/2012/08/15/
# - http://www.di.ens.fr/cmvs/

# update the system
sudo apt-get update && sudo apt-get upgrade
sudo apt-get install build-essential

# install a proprietary nvidia driver
sudo apt-get install nvidia-331-updates

# prepare dependencies of vsfm, siftgpu, mulicore_bundle_adjustment, 
#                         pmvs-2, cmvs, graclus
sudo apt-get install libgl-dev libglu-dev libx11-dev \
                     libgtk2.0-dev libglew-dev glew-utils libdevil-dev \
                     libboost-all-dev libatlas-cpp-0.6-dev libatlas-dev \
                     imagemagick libatlas3gf-base libcminpack-dev \
                     libgfortran3 libmetis-edf-dev libparmetis-dev \
                     freeglut3-dev libgsl0-dev
sudo apt-get install unzip
sudo apt-get install nvidia-cuda-toolkit

# compile vsfm (framework for 3d reconstruction from motion)
wget http://ccwu.me/vsfm/download/VisualSFM_linux_64bit.zip
unzip VisualSFM_linux_64bit.zip
cd vsfm
make 
cd ..

# compile SiftGPU (feature detection with GPU)
wget -O SiftGPU.zip http://wwwx.cs.unc.edu/~ccwu/cgi-bin/siftgpu.cgi
unzip SiftGPU.zip
cd SiftGPU
sudo ln -s /usr/lib/nvidia-cuda-toolkit /usr/local/cuda
make
cd ..
cp SiftGPU/bin/libsiftgpu.so vsfm/bin/

# pba (fast bundle adjustments library)
wget http://grail.cs.washington.edu/projects/mcba/pba_v1.0.5.zip 
unzip pba_v1.0.5.zip
cd pba
sed -i.bak 's|CUDA_INSTALL_PATH = /usr/local/cuda|CUDA_INSTALL_PATH \
           = /usr/lib/nvidia-cuda-toolkit|g' ./makefile
make
cd ..
cp pba/bin/libpba.so vsfm/bin/

# pmvs-2 (builds a point cloud of features seen in multiple views)
sudo apt-get install liblapack-dev
wget http://www.di.ens.fr/pmvs/pmvs-2-fix0.tar.gz
tar xf pmvs-2-fix0.tar.gz
cd pmvs-2/program/main/
make clean

wget http://www.10flow.com/wp-content/uploads/2012/08/mylapack.o.tar.gz
tar xf mylapack.o.tar.gz
make depend
make
cd ../../../
\end{minted}

\begin{minted}[bgcolor=tango-bg,frame=lines,framesep=2mm,samepage=true,fontsize=\footnotesize]{bash} 
# graclus (fast graph clustering library)
wget http://www.cs.utexas.edu/users/dml/Software/graclus1.2.tar.gz
tar xf graclus1.2.tar.gz
cd graclus1.2/
sed -i.bak 's|COPTIONS = -DNUMBITS=32|COPTIONS \
           = -DNUMBITS=64|g' Makefile.in
make
cd ..

# cmvs (cluster pmvs output)
wget http://www.di.ens.fr/cmvs/cmvs-fix2.tar.gz
tar xf cmvs-fix2.tar.gz
cp pmvs-2/program/main/mylapack.o cmvs/program/main/
sed -i.bak '1s/^/#include <vector>\n#include <numeric>\n/' \
           cmvs/program/base/cmvs/bundle.cc
sed -i.bak '1s/^/#include <stdlib.h>\n/' cmvs/program/main/genOption.cc
sed -i.bak 's|YOUR_INCLUDE_METIS_PATH =|YOUR_INCLUDE_METIS_PATH \
           = -I'$(pwd)'/graclus1.2/metisLib|g' \
           cmvs/program/main/Makefile
sed -i.bak 's|YOUR_LDLIB_PATH =|YOUR_LDLIB_PATH \
           = -L'$(pwd)'/graclus1.2|g' cmvs/program/main/Makefile
sed -i.bak 's/^Your\ /# Your /g' cmvs/program/main/Makefile
cd cmvs/program/main
make

cp cmvs ../../../vsfm/bin
cp pmvs2 ../../../vsfm/bin
cp genOption ../../../vsfm/bin

# set export paths
echo "" >> ~/.bashrc
echo export PATH=\$PATH:$(pwd)/vsfm/bin >> ~/.bashrc
echo export LD_LIBRARY_PATH=\$LD_LIBRARY_PATH:$(pwd)/vsfm/bin \
            >> ~/.bashrc
\end{minted}



