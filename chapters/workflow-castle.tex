\chapter{Einleitung}

%----------------------------------------------------------------------------------------

Die folgenden Kapitel beschreiben den Prozess, der nötig war um das Schloss
Rapperswil mittels Luftbilder als 3D-Modell zu rekonstruieren. Wir gehen hier
nicht auf die Evaluation der genutzten Software ein, dies folgt im nächsten Teil
dieser Dokumentation.

%----------------------------------------------------------------------------------------

\section{Erfassung des Schlosses mit Kamera-Drohne}

Um Fotos des Schlosses von allen Winkeln zu erstellen, braucht man ein
ferngesteuertes Flugobjekt, wie \zb{} einen Multikopter. In unserem Fall wurden
die Fotos von Michael Suter (\url{http://lightmoment.ch/}) erstellt, der einen
Quadrokopter besitzt und damit freundlicherweise seine Piloten-Fähigkeiten unter
Beweis gestellt hat.

\subsection{Materialliste}

\begin{itemize}
	\item Quadrokopter: \textit{Team BlackSheep Discovery
		Pro}\footnote{\url{http://www.team-blacksheep.com/products/prod:discopro}}
	\item Kamera: \textit{GoPro Hero 4 Black}\footnote{\url{https://gopro.com/}}
	\item GPS Tracker: \textit{Fairphone
		FP1}\footnote{\url{https://www.fairphone.com/}} mit
		\textit{GeoTracker}
		App\footnote{\url{https://play.google.com/store/apps/details?id=com.ilyabogdanovich.geotracker}}
\end{itemize}

\subsection{Vorgehen}

Die maximale Flugzeit pro Akku beträgt 10--15 Minuten. Wir hatten zwei geladene
Akkus dabei und erreichten somit eine Flugzeit von 20--30 Minuten.

Die GoPro Kamera hatten wir so eingestellt, dass sie zwei mal pro Sekunde ein
Bild machte. Daraus ergaben sich dann etwa 2700 Fotos, was ca. 5.3 GiB
Bildmaterial entspricht.

Um Bewegungs-Unschärfe bei den Bildern zu vermeiden, ist es wichtig, dass der
Multikopter über eine Kamera-Stabilisierung verfügt. Dies ist bei der genutzten
TBS Discovery mit einem Gimbal (gyroskopische Mehrachsen-Stabilisierung)
gegeben.

Zur Aufzeichnung der GPS-Koordinaten haben wir ein Smartphone mit einer GPS
Tracking App auf dem Quadrokopter befestigt. Das ist nicht sehr professionell,
hat aber gut funktioniert.

Mit dem Quadrokopter sind wir dann während einer halben Stunde mehrmals um das
Schloss herumgeflogen, um Fotos von jedem Winkel zu erstellen. Auch den Innehof
haben wir mit dem Quadrokopter abgeflogen. Anschliessend sind wir noch zu Fuss
mit der Kamera in der Hand dem Pfad neben dem Schloss gefolgt um auch gutes
Bildmaterial von innerhalb des Tores zu erhalten.

\vspace{1\baselineskip}

\begin{figure}[H]
	\centering
	\includegraphics[width=\textwidth]{images/gpstrace_satellite.png}
	\caption{GPS Trace des Quadrokopters während der Erfassung.\\Bildmaterial:
		\copyright{} Bing Maps}
	\label{img:gpstrace-satellite}
\end{figure}

\begin{figure}[H]
	\centering
	\includegraphics[width=\textwidth]{images/gpstrace_mapnik.png}
	\caption{GPS Trace des Quadrokopters während der Erfassung.\\Bildmaterial:
		\copyright{} OpenStreetMap Contributors}
	\label{img:gpstrace-mapnik}
\end{figure}

\subsection{Bildqualität}\label{workflow:erfassung:bildqualitaet}

Die GoPro Kamera ist für Sportaktivitäten entwickelt worden und hat daher ein
Weitwinkel-Objektiv ("<Fisheye">) eingebaut. Dies ist bestens geeignet um
beispielsweise beim Skifahren einen guten Überblick zu bewahren, für die
Rekonstruktion ist es aber eher nachteilig.

Ein weiterer Faktor ist die Kompression. Um viele Bilder auf der SD-Karte
speichern zu können, werden die JPEG-Bilder stark komprimiert. Die daraus
entstehenden JPEG-Artefakte können die Feature-Detection stören.

\marginpar{Das Raw-Format, auch Rohdaten- format genannt, bezeichnet eine Familie
von Dateiformaten für Digitalkameras, bei denen die Kamera die
Bildinfor- mationen nach der Digitalisierung direkt ohne weitere Bearbeitung auf
das Speichermedium schreibt. Bilder im Rohdatenformat werden gelegentlich auch
als "<Digitales Negativ"> bezeichnet.}

Idealerweise würde man daher zur Erfassung eine Spiegelreflex-Kamera mit
neutralem 35mm Objektiv verwenden und alle Bilder im Raw-Format abspeichern.
Diese können dann mit entsprechender Software ohne verlustbehaftete Kompression
in ein für die Photogrammetrie\-/Software nutzbares Format umgewandelt werden.

In unserem Fall waren wir aber durch die Tragfähigkeit des Quadrokopters
eingeschränkt und haben uns stattdessen dafür entschieden, die Bilder am
Computer mithilfe eines Linsenprofils in Photoshop
Lightroom\footnote{\url{http://www.photoshop.com/products/photoshoplightroom}}
zu entzerren.

Inzwischen gibt es zwar Photogrammetrie\-/Programme welche die GoPro
Linsenprofile direkt unterstützen. Nichtsdestotrotz gibt es sicher Kameras,
welche besser für solche Aufgaben geeignet sind als die GoPro.

\subsection{Learnings}

\begin{itemize}
	\item Die Jahreszeit war an sich gut gewählt, da die Bäume im Winter nicht
		belaubt sind, wodurch die Sicht auf das Schloss nicht eingeschränkt wird.
	\item Nachteilig war jedoch der Schnee auf den Dächern. Er überdeckte die
		Dachziegel und führte so bei der Feature-Erkennung zu schlechteren
		Ergebnissen.
	\item Die GoPro Kamera hat eine äusserst starke Linsen-Verzerrung. Eine Kamera
		mit neutraler Linse wäre vermutlich besser geeignet.
	\item Die JPEG Kompression ist möglicherweise störend. Eine Kamera mit
		Unterstützung für unkomprimierte Bilder (Raw-Format) würde vielleicht
		bessere Resultate liefern.
\end{itemize}

%----------------------------------------------------------------------------------------

\section{Nachbearbeiten des Bildmaterials}

Um wie im letzten Kapitel bereits besprochen, haben wir unsere Bilder mithilfe
von Photoshop Lightroom nachbearbeitet um die Linsenverzerrung herauszurechnen.

Das Linsenprofil der GoPro 4 wird seit Photoshop Lightroom 5.7 unterstützt. Die
Korrektur kann für alle Bilder gleichzeitig im Batch-Korrektur-Modus erfolgen.
Nach der Korrektur haben wir die Bilder mit JPEG-Qualitätsstufe 100 exportiert.

Als Open Source Alternative zu Photoshop Lightroom wäre noch
Hugin\footnote{\url{http://hugin.sourceforge.net/}} zu erwähnen, mit dem
ebenfalls Bilder entzerrt werden können. Wir haben jedoch kein vorbereitetes
Linsenprofil für die GoPro 4 gefunden und uns deshalb aus Zeitgründen für
Lightroom entschieden. Ein Linsenprofil könnte jedoch selber erstellt werden.
Anleitungen dazu gibt es
online\footnote{\url{http://hugin.sourceforge.net/tutorials/calibration/en.shtml}}.

TODO: https://github.com/racerxdl/goprocorrect

%----------------------------------------------------------------------------------------

\section{Erster Versuch: Pix4D}

Die erste von uns getestete Software war
Pix4D\footnote{\url{https://pix4d.com/}}. Dieses Programm wurde von einem
EFPL-Spinoff in der Westschweiz entwickelt. Während die Vollversion relativ
teuer ist (7'900 CHF), gibt es eine Testversion mit eingeschränktem
Funktionsumfang (keine GPU-Beschleunigung, kein Export), welche wir für unseren
ersten Versuch genutzt haben.

Da die Software damit wirbt, problemlos mit grossen Datenmengen umgehen zu
können, haben wir direkt mal alle 2700 Fotos hineingeladen und einen
Rekonstruktions-Prozess gestartet.

Die Rekonstruktion dauerte auf einem Intel Core i7-4790K mit 16 GiB RAM etwas
über 52 Stunden. Das resultierende Modell war jedoch leider fehlerhaft. Wie man
auf der Abbildung \ref{TODO} erkennen kann, wurde die Gasse neben dem Schloss
der Schlossmauer entlang schräg nach oben geführt. Zudem war im endgültigen
Ergebnis der Hauptturm doppelt vorhanden, einmal schräg versetzt.

TODO bild

Unsere Vermutung ist, dass wir zu viele Input-Daten verwendet hatten, so dass
sich falsche Matches so weit summiert haben dass die Software sie als gültig
betrachtet hat.

TODO learnings: ev weniger bilder, ev sequentielle algorithmen

%----------------------------------------------------------------------------------------

\section{Zweiter Versuch: VisualSFM}

TODO

%----------------------------------------------------------------------------------------

\section{Mesh-Generierung}

TODO

%----------------------------------------------------------------------------------------

\section{Mesh-Bereinigung}

TODO

%----------------------------------------------------------------------------------------

\section{3D-Druck}

TODO

%----------------------------------------------------------------------------------------

\section{Resultate}

TODO
