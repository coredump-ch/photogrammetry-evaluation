\chapter{Photogrammetrie}

\section{Was ist Photogrammetrie}

TODO

%----------------------------------------------------------------------------------------

\section{Das Bildmaterial}

Die ideale Ausgangsquelle für 3D-Rekonstruktion mittels Photogrammetrie sind
eine hohe Anzahl qualitativ hochwertiger Fotos des Zielobjektes von allen
Seiten. Jede Oberfläche sollte darin zu sehen sein. Die Bilder sollten sich
überlappen, damit die Photogrammetrie-Software daraus die ursprüngliche
Oberfläche errechnen kann. Idealerweise enthalten die Bilder auch
GSNS[TODO:?]-Koordinaten. Damit kann der Rekonstruktions-Prozess beschleunigt
werden, zudem ist nur so eine einfache Georeferenzierung möglich.

%----------------------------------------------------------------------------------------

\section{Feature-Erkennung}

Als nächster Schritt wird das Bildmaterial in eine Photogrammetrie-Software
geladen. Diese versucht nun, auf den Bildern sogenannte "<Features"> zu
erkennen. Features sind [TODO]

%----------------------------------------------------------------------------------------

\section{Sparse Point Cloud}

TODO: Übersetzen

Die [TODO übersetzung], im Englischen "<Sparse Point Cloud"> genannt, ist eine
dünn besetzte Punktwolke [WIP]
