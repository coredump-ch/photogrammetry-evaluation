\chapter{Einleitung}

\label{ch:eval-intro}

%----------------------------------------------------------------------------------------

In diesem Teil der Dokumentation vergleichen wir verschiedene
Photogrammetrie\-/Lösungen. Berücksichtigt werden folgende Produkte:

\begin{figure}[H]
	\begin{tabularx}{\textwidth}{XXl}
		\toprule
		\textbf{Name} & \textbf{Lizenz} & \textbf{Preis} \\
		\midrule
		VisualSFM & Teilweise OSS & Gratis \\
		OpenDroneMap & Freie Software & Gratis \\
		Pix4Dmapper Pro & Proprietär & 7900 CHF \\
		PhotoScan Pro & Proprietär & 3499 USD \\
		\bottomrule
	\end{tabularx}
	\caption{Evaluierte Photogrammetrie-Software: Lizenz / Kosten}
	\label{table:eval:software-license}
\end{figure}

\noindent Die Produkte bieten folgenden Funktionsumfang:

\begin{figure}[H]
	\begin{tabularx}{\textwidth}[H]{Xccc}
		\toprule
		\textbf{Name} & \textbf{3D Rekonstruktion} & \textbf{Orthofoto} & \textbf{DSM} \\
		\midrule
		VisualSFM & Ja & Nein & Jein \\
		OpenDroneMap & Nein & Ja & Ja \\
		Pix4Dmapper Pro & Ja & Ja & Ja \\
		PhotoScan Pro & Ja & Ja & Ja \\
		\bottomrule
	\end{tabularx}
	\caption{Evaluierte Photogrammetrie-Software: Funktionsumfang}
	\label{table:eval:software-capabilities}
\end{figure}

Für die Evaluation werden zwei Datasets verwendet: Ein Dataset von einem
Stroh-Hasen in Jona zum Testen der 3D-Rekonstruktion und ein Dataset mit
Senkrecht-Fotos des HSR-Geländes zum Testen der Orthofoto- und DSM-Generierung.

Verglichen werden dabei die Dauer der Rekonstruktion, die Anzahl der generierten
Punkte in der Punktwolke, die Anzahl der Vertizes im generierten Mesh und die
Resultate selbst.

Alle Tests wurden auf einem System mit einem Intel Core i7-4770 3.4 GHz
Prozessor und 32 GiB RAM durchgeführt. Für die Rekonstruktion wurden in jeder
Software die Standardeinstellungen verwendet.

%----------------------------------------------------------------------------------------
